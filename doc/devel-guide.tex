\documentclass[english]{article}
\usepackage[T1]{fontenc}
\usepackage[latin1]{inputenc}
\usepackage{alltt}
\pagestyle{headings}
\setcounter{tocdepth}{2}
\setlength\parskip{\medskipamount}
\setlength\parindent{0pt}

\makeatletter

%%%%%%%%%%%%%%%%%%%%%%%%%%%%%% LyX specific LaTeX commands.
%% Because html converters don't know tabularnewline
\providecommand{\tabularnewline}{\\}

\usepackage{babel}
\makeatother
\begin{document}

\title{Factor Developer's Guide}


\author{Slava Pestov}

\maketitle
\tableofcontents{}


\newpage
\section*{Introduction}

Factor is an imperative programming language with functional and object-oriented
influences. Its primary focus is the development of web-based server-side
applications. Factor borrows heavily from Forth, Joy and Lisp. Programmers familiar with these languages will recognize many similarities with Factor.

Factor is \emph{interactive}. This means it is possible to run a Factor interpreter that reads from the keyboard, and immediately executes expressions as they are entered. This allows words to be defined and tested one at a time.

Factor is \emph{dynamic}. This means that all objects in the language are fully reflective at run time, and that new definitions can be entered without restarting the interpreter. Factor code can be used interchangably as data, meaning that sophisticated language extensions can be realized as libraries of words.

Factor is \emph{safe}. This means all code executes in a virtual machine that provides
garbage collection and prohibits direct pointer arithmetic. There is no way to get a dangling reference by deallocating a live object, and it is not possible to corrupt memory by overwriting the bounds of an array.

When examples of interpreter interactions are given in this guide, the input is in a roman font, and any
output from the interpreter is in italics:

\begin{alltt}
"Hello, world!" print
\emph{Hello, world!}
\end{alltt}

\section{Fundamentals}

A ``word'' is the main unit of program organization
in Factor -- it corresponds to a ``function'', ``procedure''
or ``method'' in other languages.

A typical Factor development session involves a text editor and Factor
interpreter running side by side. Instead of the edit/compile/run
cycle, the development process becomes an {}``edit cycle'' -- you
make some changes to the source file and reload it in the interpreter
using a command like this:

\begin{alltt}
"numbers-game.factor" run-file
\end{alltt}

Then the changes can be tested, either by hand, or using a test harness.
There is no need to compile anything, or to lose interpreter state
by restarting. Additionally, words with {}``throw-away'' definitions
that you do not intend to keep can also be entered directly at this
interpreter prompt.

Factor emphasizes \emph{bottom-up design}. Each word should do one useful task. New words can be defined in terms
of existing, already-tested words. You design a set of reusable words
that model the problem domain. Then, the problem is solved in terms
of a \emph{domain-specific vocabulary}, and Factor programs are really just libraries of ureusable words.

\subsection{The stack}

The stack is used to exchange data between words. When a number is
executed, it is pushed on the stack. When a word is executed, it receives
input parameters by removing successive elements from the top of the
stack. Results are then pushed back to the top of the stack. 

The word \texttt{.s} prints the contents of the stack, leaving the
contents of the stack unaffected. The top of the stack is the rightmost
element in the printout:

\begin{alltt}
2 3 .s
\emph{\{ 2 3 \}}
\end{alltt}

The word \texttt{.} removes the object at the top of the stack, and
prints it:

\begin{alltt}
1 2 3 . . .
\emph{3}
\emph{2}
\emph{1}
\end{alltt}

The word \texttt{clear} removes all entries from the stack. It should only ever be used interactively, not from a definition!

\begin{alltt}
"hey ho" "merry christmas" .s
\emph{\{ "hey ho" "merry christmas" \}}
clear .s
\emph{\{ \}}
\end{alltt}

The usual arithmetic operators \texttt{+ - {*} /} all take two parameters
from the stack, and push one result back. Where the order of operands
matters (\texttt{-} and \texttt{/}), the operands are taken in the natural order. For example:

\begin{alltt}
10 17 + .
\emph{27}
111 234 - .
\emph{-123}
333 3 / .
\emph{111}
\end{alltt}

This type of arithmetic is called \emph{postfix}, because the operator
follows the operands. Contrast this with \emph{infix} notation used
in many other languages, so-called because the operator is in-between
the two operands.

More complicated infix expressions can be translated into postfix
by translating the inner-most parts first. Grouping parentheses are
never necessary:

\begin{alltt}
! Postfix equivalent of (2 + 3) {*} 6
2 3 + 6 {*}
\emph{30}
! Postfix equivalent of 2 + (3 {*} 6)
2 3 6 {*} +
\emph{20}
\end{alltt}

\subsection{Factoring}

New words can be defined in terms of existing words using the \emph{colon
definition} syntax:

\begin{alltt}
: \emph{name} ( \emph{inputs} -{}- \emph{outputs} )
    ! \emph{Description}
    \emph{factors ...} ;
\end{alltt}

When the new word is executed, each one of its factors gets executed,
in turn.The stack effect comment delimited by \texttt{(} and \texttt{)},
as well as the documentation comment starting with \texttt{!} are
both optional, and can be placed anywhere in the source code, not
just in colon definitions. The interpreter ignores comments -- don't you.

Note that in a source file, a word definition can span multiple lines.
However, the interactive interpreter expects each line of input to
be ``complete'', so colon definitions that are input interactively must contain line breaks.

For example, say we are designing some aircraft
navigation software. Suppose we need a word that takes the flight time, the aircraft
velocity, and the tailwind velocity, and returns the distance travelled.
If the parameters are given on the stack in that order, all we do
is add the top two elements (aircraft velocity, tailwind velocity)
and multiply it by the element underneath (flight time). So the definition
looks like this:

\begin{alltt}
: distance ( time aircraft tailwind -{}- distance ) + {*} ;
2 900 36 distance .
\emph{1872}
\end{alltt}

Note that we are not using any distance or time units here. To extend this example to work with units, first assume that internally, all distances are
in meters, and all time intervals are in seconds. We can define words
for converting from kilometers to meters, and hours and minutes to
seconds:

\begin{alltt}
: kilometers 1000 {*} ;
: minutes 60 {*} ;
: hours 60 {*} 60 {*} ;
2 kilometers .
\emph{2000}
10 minutes .
\emph{600}
2 hours .
\emph{7200}
\end{alltt}

The implementation of \texttt{km/hour} is a bit more complex -- to convert from kilometers per hour to our ``canonical'' meters per second, we have to first convert to kilometers per second, then divide this by the number of seconds in one hour to get the desired result:

\begin{alltt}
: km/hour kilometers 1 hours / ;
2 hours 900 km/hour 36 km/hour distance .
\emph{1872000}
\end{alltt}

\subsection{Stack effects}

A stack effect comment contains a description of inputs to the left
of \texttt{-{}-}, and a description of outputs to the right. As always,
the top of the stack is on the right side. Lets try writing a word
to compute the cube of a number.

Three numbers on the stack can be multiplied together using \texttt{{*}
{*}}:

\begin{alltt}
2 4 8 {*} {*} .
\emph{64}
\end{alltt}
However, the stack effect of \texttt{{*} {*}} is \texttt{( a b c -{}-
a{*}b{*}c )}. We would like to write a word that takes \emph{one} input
only. To achieve this, we need to be able to duplicate the top stack
element twice. As it happens, there is a word \texttt{dup ( x -{}-
x x )} for precisely this purpose. Now, we are able to define the
\texttt{cube} word:

\begin{alltt}
: cube dup dup {*} {*} ;
10 cube .
\emph{1000}
-2 cube .
\emph{-8}
\end{alltt}
It is quite often the case that we want to compose two factors in
a colon definition, but their stack effects don't {}``match up''.

There is a set of \emph{shuffle words} for solving precisely this
problem. These words are so-called because they simply rearrange stack
elements in some fashion, without modifying them in any way. Lets
take a look at the most frequently-used shuffle words:

\texttt{drop ( x -{}- )} Discard the top stack element. Used when
a word returns a value that is not needed.

\texttt{dup ( x -{}- x x )} Duplicate the top stack element. Used
when a value is required as input for more than one word.

\texttt{swap ( x y -{}- y x )} Swap top two stack elements. Used when
a word expects parameters in a different order.

\texttt{rot ( x y z -{}- y z x )} Rotate top three stack elements
to the left.

\texttt{-rot ( x y z -{}- z x y )} Rotate top three stack elements
to the right.

\texttt{over ( x y -{}- x y x )} Bring the second stack element {}``over''
the top element.

\texttt{nip ( x y -{}- y )} Remove the second stack element.

\texttt{tuck ( x y -{}- y x y )} Tuck the top stack element under
the second stack element.

\texttt{dupd ( x y -{}- x x y )} Duplicate the second stack element.

\texttt{swapd ( x y z -{}- y x z )} Swap the second and third stack elements.

\texttt{transp ( x y z -{}- z y x )} Swap the first and third stack elements.

\texttt{2drop ( x y -{}- )} Discard the top two stack elements.

\texttt{2dup ( x y -{}- x y x y )} Duplicate the top two stack elements. A frequent use for this word is when two values have to be compared using something like \texttt{=} or \texttt{<} before being passed to another word.

\texttt{2swap ( x y z t -{}- z t x y )} Swap the top two stack elements.

You should try all these words out and become familiar with them. Push some numbers on the stack,
execute a shuffle word, and look at how the stack contents was changed using
\texttt{.s}. Compare the stack contents with the stack effects above.

Note the order of the shuffle word descriptions above. The ones at
the top are used most often because they are easy to understand. The
more complex ones such as \texttt{rot} and \texttt{2swap} should be avoided unless absolutely necessary, because
they make the flow of data in a word definition harder to understand.

If you find yourself using too many shuffle words, or you're writing
a stack effect comment in the middle of a colon definition, it is
a good sign that the word should probably be factored into two or
more words. Each word should take at most a couple of sentences to describe. Effective factoring is like riding a bicycle -- once you ``get it'', it becomes second nature.


\subsection{Vocabularies}

When an expression is parsed, each token in turn is looked up in the dictionary. If there is no dictionary entry, the token is parsed as a number instead.
The dictionary of words is structured as a set of named \emph{vocabularies}. Each vocabulary is a list
of related words -- for example, the {}``lists''
vocabulary contains words for working with linked lists.

When a word is read by the parser, the \emph{vocabulary search path}
determines which vocabularies to search. In the interactive interpreter,
the default search path contains a large number of vocabularies. Contrast
this to the situation when a file is being parsed -- the search path
has a minimal set of vocabularies containing basic parsing words.%
\footnote{The rationale here is that the interactive interpreter should have
a large number of words available for convenience, whereas
source files should specify their external dependencies explicitly.%
}

How do you know which vocabulary contains a word? Vocabularies can
either be listed, and ``apropos'' searches can be performed:

\begin{alltt}
"init" words.
\emph{{[} ?run-file boot cli-arg cli-param init-environment}
\emph{init-gc init-interpreter init-scratchpad init-search-path}
\emph{init-stdio init-toplevel parse-command-line parse-switches}
\emph{run-files run-user-init stdin stdout {]} }

"map" apropos.
\emph{IN: lists}
\emph{map}
\emph{IN: strings}
\emph{str-map}
\emph{IN: vectors}
\emph{(vector-map)}
\emph{(vector-map-step)}
\emph{vector-map }
\end{alltt}

New vocabularies are added to the search path using the \texttt{USE:}
parsing word. For example:

\begin{alltt}
"/home/slava/.factor-rc" exists? .
\emph{ERROR: <interactive>:1: Undefined: exists?}
USE: streams
"/home/slava/.factor-rc" exists? .
\emph{t}
\end{alltt}

New words are defined in the \emph{input vocabulary}. The input vocabulary
can be changed at the interactive prompt, or in a source file, using
the \texttt{IN:} parsing word. For example:

\begin{alltt}
IN: music-database
: random-playlist ... ;
\end{alltt}
It is a convention (although it is not enforced by the parser) that
the \texttt{IN:} directive is the first statement in a source file,
and all \texttt{USE:} follow, before any other definitions.

Here is an example of a typical series of vocabulary declarations:

\begin{alltt}
IN: todo-list
USE: kernel
USE: lists
USE: math
USE: strings
\end{alltt}

\subsection{Booleans and logic}

Words that return a boolean truth value are known as \emph{predicates}. Predicates are usually used to decide what to execute next at branch points. In Factor, there is no special boolean data type
-- instead, a special object \texttt{f} is the only object with a
``false'' boolean value. Every other object is a boolean ``true''.
The special object \texttt{t} is the ``canonical'' truth value. Note that words that return booleans don't return \texttt{t} as a rule; any object that is not equal to \texttt{f} can be returned as the true value.

The usual boolean operations are found in the \texttt{logic} vocabulary. Note that these are not integer bitwise operations; bitwise operations are described in the next chapter.

\texttt{>boolean ( ? -{}- ? )} returns \texttt{t} if the top of stack is anything except \texttt{f}, and \texttt{f} otherwise. So it does not change the boolean value of an object, but rather it converts it to canonical form. This word is rarely used.

\texttt{not ( ? -{}- ? )} returns \texttt{t} if the top of stack is \texttt{f}, and \texttt{f} otherwise.

\texttt{and ( ? ? -{}- ? )} returns a true value if both input parameters are true.

\texttt{or ( ? ? -{}- ? )} returns a true value if at least one of the input parameters is true.

\texttt{xor ( ? ? -{}- ? )} returns a true value if exactly one of the input parameters is true.

\begin{alltt}
t t and .
\emph{t}
5 f and .
\emph{f}
f "hi" or .
\emph{"hi"}
f f or .
\emph{f}
t t xor .
\emph{f}
t f xor .
\emph{t}
\end{alltt}

\subsection{Combinators}

A quotation a list of objects that can be executed. Words that execute quotations are called \emph{combinators}. Quotations are input
using the following syntax:

\begin{alltt}
{[} 2 3 + . {]}
\end{alltt}
When input, a quotation is not executed immediately -- rather, it
is pushed on the stack. Try evaluating the following:

\begin{alltt}
{[} 1 2 3 + {*} {]} .s
\emph{\{ {[} 1 2 3 + {*} {]} \}}
call .s
\emph{\{ 5 \}}
\end{alltt}
\texttt{call} \texttt{( quot -{}- )} executes the quotation at the
top of the stack. Using \texttt{call} with a literal quotation is
useless; writing out the elements of the quotation has the same effect.
However, the \texttt{call} combinator is a building block of more
powerful combinators, since quotations can be passed around arbitrarily
and even modified before being called.

\texttt{ifte} \texttt{( cond true false -{}- )} executes either the
\texttt{true} or \texttt{false} quotations, depending on the boolean
value of \texttt{cond}. Here is an example of \texttt{ifte} usage:

\begin{alltt}
1 2 < {[} "1 is less than 2." print {]} {[} "bug!" print {]} ifte
\end{alltt}
Compare the order of parameters here with the order of parameters in
the stack effect of \texttt{ifte}.

The stack effects of the two \texttt{ifte} branches should be
the same. If they differ, the word becomes harder to document and
debug.

\texttt{when} \texttt{( cond true -{}- )} and \texttt{unless} \texttt{( cond false -{}- )} are variations of \texttt{ifte} with only one active branch. The branches should produce as many values as they consume; this ensures that the stack effect of the entire \texttt{when} or \texttt{unless} expression is consistent regardless of which branch was taken.

\texttt{times ( num quot -{}- )} executes a quotation a number of
times. It is good style to have the quotation always consume as many
values from the stack as it produces. This ensures the stack effect
of the entire \texttt{times} expression stays constant regardless
of the number of iterations.

More combinators will be introduced in later sections.

\subsection{Recursion}

The idea of \emph{recursion} is key to understanding Factor. A \emph{recursive} word definition is one that refers to itself, usually in one branch of a conditional.

FIXME

\section{Numbers}

Factor provides a rich set of math words. Factor numbers more closely model the mathematical concept of a number than other languages. Where possible, exact answers are given -- for example, adding or multiplying two integers never results in overflow, and dividing two integers yields a fraction rather than a truncated result. Complex numbers are supported, allowing many functions to be computed with parameters that would raise errors or return ``not a number'' in other languages.

\subsection{Integers}

The simplest type of number is the integer. Integers come in two varieties -- \emph{fixnums} and \emph{bignums}. As their names suggest, a fixnum is a fixed-width quantity\footnote{Fixnums range in size from $-2^{w-3}-1$ to $2^{w-3}$, where $w$ is the word size of your processor (for example, 32 bits). Usually, you do not have to worry about details like this.}, and is a bit quicker to manipulate than an arbitrary-precision bignum.

The predicate word \texttt{integer?} tests if the top of the stack is an integer. If this returns true, then exactly one of \texttt{fixnum?} or \texttt{bignum?} would return true for that object. Usually, your code does not have to worry if it is dealing with fixnums or bignums.

Unlike some languages where the programmer has to declare storage size explicitly and worry about overflow, integer operations automatically return bignums if the result would be too big to fit in a fixnum. Here is an example where multiplying two fixnums returns a bignum:

\begin{alltt}
134217728 fixnum? .
\emph{t}
128 fixnum? .
\emph{t}
134217728 128 * .
\emph{17179869184}
134217728 128 * bignum? .
\emph{t}
\end{alltt}

Integers can be entered using a different base. By default, all number entry is in base 10, however this can be changed by prefixing integer literals with one of the parsing words \texttt{BIN:}, \texttt{OCT:}, or \texttt{HEX:}. For example:

\begin{alltt}
BIN: 1110 BIN: 1 + .
\emph{15}
HEX: deadbeef 2 * .
\emph{7471857118}
\end{alltt}

The word \texttt{.} prints numbers in decimal, regardless of how they were input. A set of words in the \texttt{unparser} vocabulary is provided for turning integers into string representations in another base. These strings can then be printed using \texttt{print} from the \texttt{stdio} vocabulary.

\begin{alltt}
1234 >hex print
\emph{4d2}
1234 >bin print
\emph{10011010010}
\end{alltt}

\subsection{Rational numbers}

If we add, subtract or multiply any two integers, the result is always an integer. However, this is not the case with division. When dividing a numberator by a denominator where the numerator is not a integer multiple of the denominator, a ratio is returned instead.

\begin{alltt}
1210 11 / .
\emph{110}
100 330 / .
\emph{10/33}
\end{alltt}

Ratios are printed and can be input literally in the form of the second example. Ratios are always reduced to lowest terms by factoring out the \emph{greatest common divisor} of the numerator and denominator. A ratio with a denominator of 1 becomes an integer. Trying to create a ratio with a denominator of 0 raises an error.

The predicate word \texttt{ratio?} tests if the top of the stack is a ratio. The predicate word \texttt{rational?} returns true if and only if one of \texttt{integer?} or \texttt{ratio?} would return true for that object. So in Factor terms, a ``ratio'' is a rational number whose denominator is not equal to 1.

Ratios behave just like any other number -- all numerical operations work as expected, and in fact they use the formulas for adding, subtracting and multiplying fractions that you learned in high school.

\begin{alltt}
1/2 1/3 + .
\emph{5/6}
100 6 / 3 * .
\emph{50}
\end{alltt}

Ratios can be deconstructed into their numerator and denominator components using the \texttt{numerator} and \texttt{denominator} words. The numerator and denominator are both integers, and furthermore the denominator is always positive. When applied to integers, the numerator is the integer itself, and the denominator is 1.

\begin{alltt}
75/33 numerator .
\emph{25}
75/33 denominator .
\emph{11}
12 numerator .
\emph{12}
\end{alltt}

\subsection{Real numbers}

Rational numbers represent \emph{exact} quantities. On the other hand, a floating point number is an \emph{approximation}. While rationals can grow to any required precision, floating point numbers are fixed-width, and manipulating them is usually faster than manipulating ratios or bignums (but slower than manipulating fixnums). Floating point literals are often used to represent irrational numbers, which have no exact representation as a ratio of two integers. Floating point literals are input with a decimal point.

\begin{alltt}
1.23 1.5 + .
\emph{1.73}
\end{alltt}

The predicate word \texttt{float?} tests if the top of the stack is a floating point number. The predicate word \texttt{real?} returns true if and only if one of \texttt{rational?} or \texttt{float?} would return true for that object.

Floating point numbers are \emph{contagious} -- introducing a floating point number in a computation ensures the result is also floating point.

\begin{alltt}
5/4 1/2 + .
\emph{7/4}
5/4 0.5 + .
\emph{1.75}
\end{alltt}

Apart from contaigion, there are two ways of obtaining a floating point result from a computation; the word \texttt{>float ( n -{}- f)} converts a rational number into its floating point approximation, and the word \texttt{/f ( x y -{}- x/y)} returns the floating point approximation of a quotient of two numbers.

\begin{alltt}
7 4 / >float .
\emph{1.75}
7 4 /f .
\emph{1.75}
\end{alltt}

Indeed, the word \texttt{/f} could be defined as follows:

\begin{alltt}
: /f / >float ;
\end{alltt}

However, the actual definition is slightly more efficient, since it computes the floating point result directly.

\subsection{Complex numbers}

Just like we had to widen the integers to the rationals in order to divide, we have to widen the real numbers to the set of \emph{complex numbers} to solve certain kinds of equations. For example, the equation $x^2 + 1 = 0$ has no solution for real $x$, because there is no real number that is a square root of -1. This is so because the real numbers are not \emph{algebraically complete}. 

Complex numbers, however, are algebraically complete, and Factor will find one solution to this equation\footnote{The other, of course being \texttt{\#\{ 0 -1 \}}.}:

\begin{alltt}
-1 sqrt .
\emph{\#\{ 0 1 \}}
\end{alltt}

The literal syntax for a complex number is \texttt{\#\{ re im \}}, where \texttt{re} is the real part and \texttt{im} is the imaginary part. For example, the literal \texttt{\#\{ 1/2 1/3 \}} corresponds to the complex number $1/2 + 1/3i$.

The words \texttt{i} an \texttt{-i} push the literals \texttt{\#\{ 0 1 \}} and \texttt{\#\{ 0 -1 \}}, respectively.

The predicate word \texttt{complex?} tests if the top of the stack is a complex number. Note that unlike math, where all real numbers are also complex numbers, Factor only considers a number to be a complex number if its imaginary part is non-zero.

Complex numbers can be deconstructed into their real and imaginary components using the \texttt{real} and \texttt{imaginary} words. Both components can be pushed at once using the word \texttt{>rect ( z -{}- re im )}.

\begin{alltt}
-1 sqrt real .
\emph{0}
-1 sqrt imaginary .
\emph{1}
-1 sqrt sqrt >rect .s
\emph{\{ 0.7071067811865476 0.7071067811865475 \}}
\end{alltt}

A complex number can be constructed from a real and imaginary component on the stack using the word \texttt{rect> ( re im -{}- z )}.

\begin{alltt}
1/3 5 rect> .
\emph{\#\{ 1/3 5 \}}
\end{alltt}

Complex numbers are stored in \emph{rectangular form} as a real/imaginary component pair (this is where the names \texttt{>rect} and \texttt{rect>} come from). An alternative complex number representation is \emph{polar form}, consisting of an absolute value and argument. The absolute value and argument can be computed using the words \texttt{abs} and \texttt{arg}, and both can be pushed at once using \texttt{>polar ( z -{}- abs arg )}.

\begin{alltt}
5.3 abs .
\emph{5.3}
i arg .
\emph{1.570796326794897}
\#\{ 4 5 \} >polar .s
\emph{\{ 6.403124237432849 0.8960553845713439 \}}
\end{alltt}

A new complex number can be created from an absolute value and argument using \texttt{polar> ( abs arg -{}- z )}.

\begin{alltt}
1 pi polar> .
\emph{\#\{ -1.0 1.224606353822377e-16 \}}
\end{alltt}

\subsection{Transcedential functions}

The \texttt{math} vocabulary provides a rich library of mathematical functions that covers exponentiation, logarithms, trigonometry, and hyperbolic functions. All functions accept and return complex number arguments where appropriate. These functions all return floating point values, or complex numbers whose real and imaginary components are floating point values.

\texttt{\^ ( x y -- x\^y )} raises \texttt{x} to the power of \texttt{y}. In the cases of \texttt{y} being equal to $1/2$, -1, or 2, respectively, the words \texttt{sqrt}, \texttt{recip} and \texttt{sq} can be used instead.

\begin{alltt}
2 4 \^ .
\emph{16.0}
i i \^ .
\emph{0.2078795763507619}
\end{alltt}

\texttt{exp ( x -- e\^x )} raises the number $e$ to a specified power. The number $e$ can be pushed on the stack with the \texttt{e} word, so \texttt{exp} could have been defined as follows:

\begin{alltt}
: exp ( x -- e\^x ) e swap \^ ;
\end{alltt}

However, it is actually defined otherwise, for efficiency.\footnote{In fact, the word \texttt{\^} is actually defined in terms of \texttt{exp}, to correctly handle complex number arguments.}

\texttt{log ( x -- y )} computes the natural (base $e$) logarithm. This is the inverse of the \texttt{exp} function.

\begin{alltt}
-1 log .
\emph{\#\{ 0.0 3.141592653589793 \}}
e log .
\emph{1.0}
\end{alltt}

\texttt{sin ( x -- y )}, \texttt{cos ( x -- y )} and \texttt{tan ( x -- y )} are the familiar trigonometric functions, and \texttt{asin ( x -- y )}, \texttt{acos ( x -- y )} and \texttt{atan ( x -- y )} are their inverses.

The reciprocals of the sine, cosine and tangent are defined as \texttt{sec}, \texttt{cosec} and \texttt{cot}, respectively. Their inverses are \texttt{asec}, \texttt{acosec} and \texttt{acot}.

\texttt{sinh ( x -- y )}, \texttt{cosh ( x -- y )} and \texttt{tanh ( x -- y )} are the hyperbolic functions, and \texttt{asinh ( x -- y )}, \texttt{acosh ( x -- y )} and \texttt{atanh ( x -- y )} are their inverses.

Similarly, the reciprocals of the hyperbolic functions are defined as \texttt{sech}, \texttt{cosech} and \texttt{coth}, respectively. Their inverses are \texttt{asech}, \texttt{acosech} and \texttt{acoth}.

\subsection{Modular arithmetic}

In addition to the standard division operator \texttt{/}, there are a few related functions that are useful when working with integers.

\texttt{/i ( x y -{}- x\%y )} performs a truncating integer division. It could have been defined as follows:

\begin{alltt}
: /i / >integer ;
\end{alltt}

However, the actual definition is a bit more efficient than that.

\texttt{mod ( x y -{}- x\%y )} computes the remainder of dividing \texttt{x} by \texttt{y}. If the result is 0, then \texttt{x} is a multiple of \texttt{y}.

\texttt{/mod ( x y -{}- x/y x\%y )} pushes both the quotient and remainder.

\begin{alltt}
100 3 mod .
\emph{1}
-546 34 mod .
\emph{-2}
\end{alltt}

\texttt{gcd ( x y -- z )} pushes the greatest common divisor of two integers; that is, a common factor, or alternatively, the largest number that both integers could be divided by and still yield integers as results. This word is used behind the scenes to reduce rational numbers to lowest terms when doing ratio arithmetic.

\subsection{Bitwise operations}

There are two ways of looking at an integer -- as a mathematical entity, or as a string of bits. The latter representation faciliates the so-called \emph{bitwise operations}.

\texttt{bitand ( x y -{}- x\&y )} returns a new integer where each bit is set if and only if the corresponding bit is set in both $x$ and $y$. If you're considering an integer as a sequence of bit flags, taking the bitwise-and with a mask switches off all flags that are not explicitly set in the mask.

\begin{alltt}
BIN: 101 BIN: 10 bitand >bin print
\emph{0}
BIN: 110 BIN: 10 bitand >bin print
\emph{10}
\end{alltt}

\texttt{bitor ( x y -{}- x|y )} returns a new integer where each bit is set if and only if the corresponding bit is set in at least one of $x$ or $y$. If you're considering an integer as a sequence of bit flags, taking the bitwise-or with a mask switches on all flags that are set in the mask.

\begin{alltt}
BIN: 101 BIN: 10 bitor >bin print
\emph{111}
BIN: 110 BIN: 10 bitor >bin print
\emph{110}
\end{alltt}

\texttt{bitxor ( x y -{}- x\^y )} returns a new integer where each bit is set if and only if the corresponding bit is set in exactly one of $x$ or $y$. If you're considering an integer as a sequence of bit flags, taking the bitwise-xor with a mask toggles on all flags that are set in the mask.

\begin{alltt}
BIN: 101 BIN: 10 bitxor >bin print
\emph{111}
BIN: 110 BIN: 10 bitxor >bin print
\emph{100}
\end{alltt}

\texttt{shift ( x n -{}- y )} returns a new integer consisting of the bits of the first integer, shifted to the left by $n$ positions. If $n$ is negative, the bits are shifted to the right instead, and bits that ``fall off'' are discarded.

\begin{alltt}
BIN: 101 5 shift >bin print
\emph{10100000}
BIN: 11111 -2 shift >bin print
\emph{111}
\end{alltt}

The attentive reader will notice that shifting to the left is equivalent to multiplying by a power of two, and shifting to the right is equivalent to performing a truncating division by a power of two.

\section{PRACTICAL: Numbers game}

In this section, basic input/output and flow control is introduced.
We construct a program that repeatedly prompts the user to guess a
number -- they are informed if their guess is correct, too low, or
too high. The game ends on a correct guess.

\begin{alltt}
numbers-game
\emph{I'm thinking of a number between 0 and 100.}
\emph{Enter your guess:} 25
\emph{Too low}
\emph{Enter your guess:} 38
\emph{Too high}
\emph{Enter your guess:} 31
\emph{Correct - you win!}
\end{alltt}

\subsection{Getting started}

Start a text editor and create a file named \texttt{numbers-game.factor}.

Write a short comment at the top of the file. Two examples of commenting style supported by Factor:

\begin{alltt}
! Numbers game.
( The great numbers game )
\end{alltt}

It is always a good idea to comment your code. Try to write simple
code that does not need detailed comments to describe; similarly,
avoid redundant comments. These two principles are hard to quantify
in a concrete way, and will become more clear as your skills with
Factor increase.

We will be defining new words in the \texttt{numbers-game} vocabulary; add
an \texttt{IN:} statement at the top of the source file:

\begin{alltt}
IN: numbers-game
\end{alltt}
Also in order to be able to test the words, issue a \texttt{USE:}
statement in the interactive interpreter:

\begin{alltt}
USE: numbers-game
\end{alltt}
This section will develop the numbers game in an incremental fashion.
After each addition, issue a command like the following to load the
source file into the Factor interpreter:

\begin{alltt}
"numbers-game.factor" run-file
\end{alltt}

\subsection{Reading a number from the keyboard}

A fundamental operation required for the numbers game is to be able
to read a number from the keyboard. The \texttt{read} word \texttt{(
-{}- str )} reads a line of input and pushes it on the stack.
The \texttt{parse-number} word \texttt{( str -{}- n )} turns a decimal
string representation of an integer into the integer itself. These
two words can be combined into a single colon definition:

\begin{alltt}
: read-number ( -{}- n ) read parse-number ;
\end{alltt}
You should add this definition to the source file, and try loading
the file into the interpreter. As you will soon see, this raises an
error! The problem is that the two words \texttt{read} and \texttt{parse-number}
are not part of the default, minimal, vocabulary search path used
when reading files. The solution is to use \texttt{apropos.} to find
out which vocabularies contain those words, and add the appropriate
USE: statements to the source file:

\begin{alltt}
USE: parser
USE: stdio
\end{alltt}
After adding the above two statements, the file should now parse,
and testing should confirm that the \texttt{read-number} word works correctly.%
\footnote{There is the possibility of an invalid number being entered at the
keyboard. In this case, \texttt{parse-number} returns \texttt{f},
the boolean false value. For the sake of simplicity, we ignore this
case in the numbers game example. However, proper error handling is
an essential part of any large program and is covered later.%
}


\subsection{Printing some messages}

Now we need to make some words for printing various messages. They
are given here without further ado:

\begin{alltt}
: guess-banner
    "I'm thinking of a number between 0 and 100." print ;
: guess-prompt "Enter your guess: " write ;
: too-high "Too high" print ;
: too-low "Too low" print ;
: correct "Correct - you win!" print ;
\end{alltt}
Note that in the above, stack effect comments are omitted, since they
are obvious from context. You should ensure the words work correctly
after loading the source file into the interpreter.


\subsection{Taking action based on a guess}

The next logical step is to write a word \texttt{judge-guess} that
takes the user's guess along with the actual number to be guessed,
and prints one of the messages \texttt{too-high}, \texttt{too-low},
or \texttt{correct}. This word will also push a boolean flag, indicating
if the game should continue or not -- in the case of a correct guess,
the game does not continue.

This description of judge-guess is a mouthful -- and it suggests that
it may be best to split it into two words. So the first word we write
handles the more specific case of an \emph{inexact} guess -- so it
prints either \texttt{too-low} or \texttt{too-high}.

\begin{alltt}
: inexact-guess ( actual guess -{}- )
     < {[} too-high {]} {[} too-low {]} ifte ;
\end{alltt}
Note that the word gives incorrect output if the two parameters are
equal. However, it will never be called this way.

With this out of the way, the implementation of judge-guess is an
easy task to tackle. Using the words \texttt{inexact-guess}, \texttt{2dup}, \texttt{2drop} and \texttt{=}, we can write:

\begin{alltt}
: judge-guess ( actual guess -{}- ? )
    2dup = {[}
        2drop correct f
    {]} {[}
        inexact-guess t
    {]} ifte ;
\end{alltt}

The word \texttt{=} is found in the \texttt{kernel} vocabulary, and the words \texttt{2dup} and \texttt{2drop} are found in the \texttt{stack} vocabulary. Since \texttt{=}
consumes both its parameters, we must first duplicate them with \texttt{2dup}. The word \texttt{correct} does not need to do anything with these two numbers, so they are popped off the stack using \texttt{2drop}. Try evaluating the following
in the interpreter to see what's going on:

\begin{alltt}
clear 1 2 2dup = .s
\emph{\{ 1 2 f \}}
clear 4 4 2dup = .s
\emph{\{ 4 4 t \}}
\end{alltt}

Test \texttt{judge-guess} with a few inputs:

\begin{alltt}
1 10 judge-guess .
\emph{Too low}
\emph{t}
89 43 judge-guess .
\emph{Too high}
\emph{t}
64 64 judge-guess .
\emph{Correct}
\emph{f}
\end{alltt}

\subsection{Generating random numbers}

The \texttt{random-int} word \texttt{( min max -{}- n )} pushes a
random number in a specified range. The range is inclusive, so both
the minimum and maximum indexes are candidate random numbers. Use
\texttt{apropos.} to determine that this word is in the \texttt{random}
vocabulary. For the purposes of this game, random numbers will be
in the range of 0 to 100, so we can define a word that generates a
random number in the range of 0 to 100:

\begin{alltt}
: number-to-guess ( -{}- n ) 0 100 random-int ;
\end{alltt}
Add the word definition to the source file, along with the appropriate
\texttt{USE:} statement. Load the source file in the interpreter,
and confirm that the word functions correctly, and that its stack
effect comment is accurate.


\subsection{The game loop}

The game loop consists of repeated calls to \texttt{guess-prompt},
\texttt{read-number} and \texttt{judge-guess}. If \texttt{judge-guess}
pushes \texttt{f}, the loop stops, otherwise it continues. This is
realized with a recursive implementation:

\begin{alltt}
: numbers-game-loop ( actual -{}- )
    dup guess-prompt read-number judge-guess {[}
        numbers-game-loop
    {]} {[}
        drop
    {]} ifte ;
\end{alltt}
In Factor, tail-recursive words consume a bounded amount of call stack
space. This means you are free to pick recursion or iteration based
on their own merits when solving a problem. In many other languages,
the usefulness of recursion is severely limited by the lack of tail-recursive
call optimization.


\subsection{Finishing off}

The last task is to combine everything into the main \texttt{numbers-game}
word. This is easier than it seems:

\begin{alltt}
: numbers-game number-to-guess numbers-game-loop ;
\end{alltt}
Try it out! Simply invoke the \texttt{numbers-game} word in the interpreter.
It should work flawlessly, assuming you tested each component of this
design incrementally!


\subsection{The complete program}

\begin{verbatim}
! Numbers game example

IN: numbers-game
USE: kernel
USE: math
USE: parser
USE: random
USE: stdio
USE: stack

: read-number ( -- n ) read parse-number ;

: guess-banner
    "I'm thinking of a number between 0 and 100." print ;
: guess-prompt "Enter your guess: " write ;
: too-high "Too high" print ;
: too-low "Too low" print ;
: correct "Correct - you win!" print ;

: inexact-guess ( actual guess -- )
     < [ too-high ] [ too-low ] ifte ;

: judge-guess ( actual guess -- ? )
    2dup = [
        2drop correct f
    ] [
        inexact-guess t
    ] ifte ;

: number-to-guess ( -- n ) 0 100 random-int ;

: numbers-game-loop ( actual -- )
    dup guess-prompt read-number judge-guess [
        numbers-game-loop
    ] [
        drop
    ] ifte ;

: numbers-game number-to-guess numbers-game-loop ;
\end{verbatim}

\section{Lists}

A list of objects is realized as a set of pairs; each pair holds a list element,
and a reference to the next pair. All words relating to cons cells and lists are found in the \texttt{lists}
vocabulary.  Lists have the following literal
syntax:

\begin{alltt}
{[} "CEO" 5 "CFO" -4 f {]}
\end{alltt}
Before we continue, it is important to understand the role of data
types in Factor. Lets make a distinction between two categories of
data types:

\begin{itemize}
\item Representational type -- this refers to the form of the data in the
interpreter. Representational types include integers, strings, and
vectors. Representational types are checked at run time -- attempting
to multiply two strings, for example, will yield an error.
\item Intentional type -- this refers to the meaning of the data within
the problem domain. This could be a length measured in inches, or
a string naming a file, or a list of objects in a room in a game.
It is up to the programmer to check intentional types -- Factor won't
prevent you from adding two integers representing a distance and a
time, even though the result is meaningless.
\end{itemize}

\subsection{Cons cells}

It may surprise you that in Factor, \emph{lists are intentional types}.
This means that they are not an inherent feature of the interpreter;
rather, they are built from a simpler data type, the \emph{cons cell}.

A cons cell is an object that holds a reference to two other objects.
The order of the two objects matters -- the first is called the \emph{car},
the second is called the \emph{cdr}.

The words \texttt{cons}, \texttt{car} and \texttt{cdr}%
\footnote{These infamous names originate from the Lisp language. Originally,
{}``Lisp'' stood for {}``List Processing''.%
} construct and deconstruct cons cells:

\begin{alltt}
1 2 cons .
\emph{{[} 1 | 2 {]}}
3 4 cons car .
\emph{3}
5 6 cons cdr .
\emph{6}
\end{alltt}
The output of the first expression suggests a literal syntax for cons
cells:

\begin{alltt}
{[} 10 | 20 {]} cdr .
\emph{20}
{[} "first" | {[} "second" | f {]} {]} car .
\emph{"first"}
{[} "first" | {[} "second" | f {]} {]} cdr car .
\emph{"second"}
\end{alltt}
The last two examples make it clear how nested cons cells represent
a list. Since this {}``nested cons cell'' syntax is extremely cumbersome,
the parser provides an easier way:

\begin{alltt}
{[} 1 2 3 4 {]} cdr cdr car .
\emph{3}
\end{alltt}
A \emph{proper list} is a set of cons cells linked by their cdr, where the last cons cell has a cdr set to \texttt{f}. Also, the object \texttt{f} by itself
is a proper list, and in fact it is equivalent to the empty list \texttt{{[}
{]}}. An \emph{improper list} is a set of cons cells that does not terminate with \texttt{f}. Improper lists are input with the following syntax:

\begin{verbatim}
[ 1 2 3 | 4 ]
\end{verbatim}

The \texttt{list?} word tests if the object at the top of the stack
is a proper list:

\begin{alltt}
"hello" list? .
\emph{f}
{[} "first" "second" | "third" {]} list? .
\emph{f}
{[} "first" "second" "third" {]} list? .
\emph{t}
\end{alltt}

It is worth mentioning a few words closely related to and defined in terms of \texttt{cons}, \texttt{car} and \texttt{cdr}.

\texttt{swons ( cdr car -{}- cons )} constructs a cons cell, with the argument order reversed. Usually, it is considered bad practice to define two words that only differ by parameter order, however cons cells are constructed about equally frequently with both orders. Of course, \texttt{swons} is defined as follows:

\begin{alltt}
: swons swap cons ;
\end{alltt}

\texttt{uncons ( cons -{}- car cdr )} pushes both constituents of a cons cell. It is defined as thus:

\begin{alltt}
: uncons dup car swap cdr ;
\end{alltt}

\texttt{unswons ( cons -{}- cdr car)} is just a swapped version of \texttt{uncons}. It is defined as thus:

\begin{alltt}
: unswons dup cdr swap car ;
\end{alltt}

\subsection{Working with lists}

Unless otherwise documented, list manipulation words expect proper
lists as arguments. Given an improper list, they will either raise
an error, or disregard the hanging cdr at the end of the list.

Also unless otherwise documented, list manipulation words return newly-created
lists only. The original parameters are not modified. This may seem
inefficient, however the absence of side effects makes code much easier
to test and debug.%
\footnote{Side effect-free code is the fundamental idea underlying functional
programming languages. While Factor allows side effects and is not
a functional programming language, for a lot of problems, coding in
a functional style gives the most maintainable and readable results.%
} Where performance is important, a set of {}``destructive'' words
is provided. They are documented in \ref{sub:Destructively-modifying-lists}.

\texttt{add ( list obj -{}- list )} Create a new list consisting of
the original list, and a new element added at the end:

\begin{alltt}
{[} 1 2 3 {]} 4 add .
\emph{{[} 1 2 3 4 {]}}
1 {[} 2 3 4 {]} cons .
\emph{{[} 1 2 3 4 {]}}
\end{alltt}
While \texttt{cons} and \texttt{add} appear to have similar effects,
they are quite different -- \texttt{cons} is a very cheap operation,
while \texttt{add} has to copy the entire list first! If you need to add to the end of a sequence frequently, consider either using a vector, or adding to the beginning of a list and reversing the list when done. For information about lists, see \ref{sub:Vectors}.

\texttt{append ( list list -{}- list )} Append two lists at the
top of the stack:

\begin{alltt}
{[} 1 2 3 {]} {[} 4 5 6 {]} append .
\emph{{[} 1 2 3 4 5 6 {]}}
{[} 1 2 3 {]} dup {[} 4 5 6 {]} append .s
\emph{\{ {[} 1 2 3 {]} {[} 1 2 3 4 5 6 {]} \}}
\end{alltt}
The first list is copied, and the cdr of its last cons cell is set
to point to the second list. The second example above shows that the original
parameter was not modified. Interestingly, if the second parameter
is not a proper list, \texttt{append} returns an improper list:

\begin{alltt}
{[} 1 2 3 {]} 4 append .
\emph{{[} 1 2 3 | 4 {]}}
\end{alltt}
\texttt{length ( list -{}- n )} Iterate down the cdr of the list until
it reaches \texttt{f}, counting the number of elements in the list:

\begin{alltt}
{[} {[} 1 2 {]} {[} 3 4 {]} 5 {]} length .
\emph{3}
{[} {[} {[} "Hey" {]} 5 {]} length .
\emph{2}
\end{alltt}
\texttt{nth ( index list -{}- obj )} Look up an element specified
by a zero-based index, by successively iterating down the cdr of the
list:

\begin{alltt}
1 {[} "Hamster" "Bagpipe" "Beam" {]} nth .
\emph{"Bagpipe"}
\end{alltt}
This word runs in linear time proportional to the list index. If you
need constant time lookups, use a vector instead.

\texttt{set-nth ( value index list -{}- list )} Create a new list,
identical to the original list except the element at the specified
index is replaced:

\begin{alltt}
{}``Done'' 1 {[} {}``Not started'' {}``Incomplete'' {]} set-nth .

\emph{{[} {}``Done'' {}``Incomplete'' {]}}
\end{alltt}
\texttt{remove ( obj list -{}- list )} Push a new list, with all occurrences
of the object removed. All other elements are in the same order:

\begin{alltt}
: australia- ( list -- list ) "Australia" swap remove ;
{[} "Canada" "New Zealand" "Australia" "Russia" {]} australia- .
\emph{{[} "Canada" "New Zealand" "Russia" {]}}
\end{alltt}
\texttt{remove-nth ( index list -{}- list )} Push a new list, with
an index removed:

\begin{alltt}
: remove-1 ( list -- list ) 1 swap remove-nth ;
{[} "Canada" "New Zealand" "Australia" "Russia" {]} remove-1 .
\emph{{[} "Canada" "Australia" "Russia" {]}}
\end{alltt}
\texttt{reverse ( list -{}- list )} Push a new list which has the
same elements as the original one, but in reverse order:

\begin{alltt}
{[} 4 3 2 1 {]} reverse .
\emph{{[} 1 2 3 4 {]}}
\end{alltt}
\texttt{contains ( obj list -{}- list )} Look for an occurrence of
an object in a list. The remainder of the list starting from the first
occurrence is returned. If the object does not occur in the list,
f is returned:

\begin{alltt}
: lived-in? ( country -{}- ? )
    {[}
        "Canada" "New Zealand" "Australia" "Russia"
    {]} contains ;
"Australia" lived-in? .
\emph{{[} "Australia" "Russia" {]}}
"Pakistan" lived-in? .
\emph{f}
\end{alltt}
For now, assume {}``occurs'' means {}``contains an object that
looks like''. The issue of object equality is covered later.

\texttt{unique ( list -{}- list )} Return a new list with all duplicate
elements removed. This word executes in quadratic time, so should
not be used with large lists. For example:

\begin{alltt}
{[} 1 2 1 4 1 8 {]} unique .
\emph{{[} 1 2 4 8 {]}}
\end{alltt}
\texttt{unit ( obj -{}- list )} Make a list of one element:

\begin{alltt}
{}``Unit 18'' unit .
\emph{{[} {}``Unit 18'' {]}}
\end{alltt}

\subsection{Association lists}

An \emph{association list} is one where every element is a cons. The
car of each cons is a name, the cdr is a value. The literal notation
is suggestive:

\begin{alltt}
{[}
    {[} "Jill"  | "CEO" {]}
    {[} "Jeff"  | "manager" {]}
    {[} "James" | "lowly web designer" {]}
{]}
\end{alltt}
\texttt{assoc? ( obj -{}- ? )} returns \texttt{t} if the object is
a list whose every element is a cons; otherwise it returns \texttt{f}.

\texttt{assoc ( key alist -{}- value )} looks for a pair with this
key in the list, and pushes the cdr of the pair. Pushes f if no pair
with this key is present. Note that \texttt{assoc} cannot differentiate between
a key that is not present at all, or a key with a value of \texttt{f}.

\texttt{assoc{*} ( key alist -{}- {[} key | value {]} )} looks for
a pair with this key, and pushes the pair itself. Unlike \texttt{assoc},
\texttt{assoc{*}} returns different values in the cases of a value
set to \texttt{f}, or an undefined value.

\texttt{set-assoc ( value key alist -{}- alist )} removes any existing
occurrence of a key from the list, and adds a new pair. This creates
a new list, the original is unaffected.

\texttt{acons ( value key alist -{}- alist )} is slightly faster
than \texttt{set-assoc} since it simply conses a new pair onto the
list. However, if used repeatedly, the list will grow to contain a
lot of {}``shadowed'' pairs.

Searching association lists incurs a linear time cost, so they should
only be used for small mappings -- a typical use is a mapping of half
a dozen entries or so, specified literally in source. Hashtables offer
better performance with larger mappings.


\subsection{List combinators}

In a traditional language such as C, every iteration or collection
must be written out as a loop, with setting up and updating of indexes,
etc. Factor on the other hand relies on combinators and quotations
to avoid duplicating these loop ``design patterns'' throughout
the code.

The simplest case is iterating through each element of a list, and
printing it or otherwise consuming it from the stack.

\texttt{each ( list quot -{}- )} pushes each element of the list in
turn, and executes the quotation. The list and quotation are not on
the stack when the quotation is executed. This allows a powerful idiom
where the quotation makes a copy of a value on the stack, and consumes
it along with the list element. In fact, this idiom works with all
well-designed combinators.%
\footnote{Later, you will learn how to apply it when designing your own combinators.%
}

The previously-mentioned \texttt{reverse} word is implemented using
\texttt{each}:
\begin{alltt}
: reverse ( list -- list ) {[} {]} swap {[} swons {]} each ;
\end{alltt}
To understand how it works, consider that each element of the original
list is consed onto the beginning of a new list, in turn. So the last
element of the original list ends up at the beginning of the new list.

\texttt{inject ( list quot -{}- list )} is similar to \texttt{each},
except after each iteration the return value of the quotation is collected into a new
list. The quotation must have stack effect
\texttt{( obj -{}- obj )} otherwise the combinator
will not function properly.

For example, suppose we have a list where each element stores the
quantity of a some nutrient in 100 grams of food; we would like to
find out the total nutrients contained in 300 grams:

\begin{alltt}
: multiply-each ( n list -{}- list )
    {[} dupd {*} {]} inject nip ;
3 {[} 50 450 101 {]} multiply-each .
\emph{{[} 180 1350 303 {]}}
\end{alltt}
Note the use of \texttt{dupd} to preserve the value of \texttt{n} after each iteration, and the final \texttt{nip} to discard the value of \texttt{n}.

\texttt{subset ( list quot -{}- list )} produces a new list containing
some of the elements of the original list. Which elements to collect
is determined by the quotation -- the quotation is called with each
list element on the stack in turn, and those elements for which the
quotation does not return \texttt{f} are added to the new list. The
quotation must have stack effect \texttt{( obj -{}- ?~)}.

For example, lets construct a list of all numbers between 0 and 99
such that the sum of their digits is less than 10:

\begin{alltt}
: sum-of-digits ( n -{}- n ) 10 /mod + ;
100 count {[} sum-of-digits 10 < {]} subset .
\emph{{[} 0 1 2 3 4 5 6 7 8 9 10 11 12 13 14 15 16 17 18 20 21}
\emph{22 23 24 25 26 27 30 31 32 33 34 35 36 40 41 42 43 44}
\emph{45 50 51 52 53 54 60 61 62 63 70 71 72 80 81 90 {]} }
\end{alltt}
\texttt{all? ( list quot -{}- ?~)} returns \texttt{t} if the quotation
returns \texttt{t} for all elements of the list, otherwise it returns
\texttt{f}. In other words, if \texttt{all?} returns \texttt{t}, then
\texttt{subset} applied to the same list and quotation would return
the entire list.%
\footnote{Barring any side effects which modify the execution of the quotation.
It is best to avoid side effects when using list combinators.%
}

For example, the implementation of \texttt{assoc?} uses \texttt{all?}:

\begin{alltt}
: assoc? ( list -{}- ?~)
    dup list? {[} {[} cons? {]} all? {]} {[} drop f {]} ifte ;
\end{alltt}

\subsection{\label{sub:List-constructors}List constructors}

The list construction words provide an alternative way to build up a list. Instead of passing a partial list around on the stack as it is built, they store the partial list in a variable. This reduces the number
of stack elements that have to be juggled.

The word \texttt{{[}, ( -{}- )} begins list construction.

The word \texttt{, ( obj -{}- )} appends an object to the partial
list.

The word \texttt{,{]} ( -{}- list )} pushes the complete list.

While variables haven't been described yet, keep in mind that a new
scope is created between \texttt{{[},} and \texttt{,{]}}. This means
that list constructions can be nested, as long as in the end, the
number of \texttt{{[},} and \texttt{,{]}} balances out. There is no
requirement that \texttt{{[},} and \texttt{,{]}} appear in the same
word, however, debugging becomes prohibitively difficult when a list
construction begins in one word and ends with another.

Here is an example of list construction using this technique:

\begin{alltt}
{[}, 1 10 {[} 2 {*} dup , {]} times drop ,{]} .
\emph{{[} 2 4 8 16 32 64 128 256 512 1024 {]}}
\end{alltt}

\subsection{\label{sub:Destructively-modifying-lists}Destructively modifying lists}

All previously discussed list modification functions always returned
newly-allocated lists. Destructive list manipulation functions on
the other hand reuse the cons cells of their input lists, and hence
avoid memory allocation.

Only ever destructively change lists you do not intend to reuse again.
You should not rely on the side effects -- they are unpredictable.
It is wrong to think that destructive words {}``modify'' the original
list -- rather, think of them as returning a new list, just like the
normal versions of the words, with the added caveat that the original
list must not be used again.

\texttt{nreverse ( list -{}- list )} reverses a list without consing.
In the following example, the return value has reused the cons cells of
the original list, and the original list has been destroyed:

\begin{alltt}
{[} 1 2 3 4 {]} dup nreverse .s
\emph{\{ {[} 1 {]} {[} 4 3 2 1 {]} \}}
\end{alltt}
Compare the second stack element (which is what remains of the original
list) and the top stack element (the list returned by \texttt{nreverse}).

The \texttt{nreverse} word is the most frequently used destructive
list manipulator. The usual idiom is a loop where values are consed
onto the beginning of a list in each iteration of a loop, then the
list is reversed at the end. Since the original list is never used
again, \texttt{nreverse} can safely be used here.

\texttt{nappend ( list list -{}- list )} sets the cdr of the last
cons cell in the first list to the second list, unless the first list
is \texttt{f}, in which case it simply returns the second list. Again,
the side effects on the first list are unpredictable -- if it is \texttt{f},
it is unchanged, otherwise, it is equal to the return value:

\begin{alltt}
{[} 1 2 {]} {[} 3 4 {]} nappend .
\emph{{[} 1 2 3 4 {]}}
\end{alltt}
Note in the above examples, we use literal list parameters to \texttt{nreverse}
and \texttt{nappend}. This is actually a very bad idea, since the same literal
list may be used more than once! For example, lets make a colon definition:

\begin{alltt}
: very-bad-idea {[} 1 2 3 4 {]} nreverse ;
very-bad-idea .
\emph{{[} 4 3 2 1 {]}}
very-bad-idea .
\emph{{[} 4 {]}}
{}``very-bad-idea'' see
\emph{: very-bad-idea}
 \emph{   {[} 4 {]} nreverse ;}
\end{alltt}
As you can see, the word definition itself was ruined!

Sometimes it is desirable make a copy of a list, so that the copy
may be safely side-effected later.

\texttt{clone-list ( list -{}- list )} pushes a new list containing
the exact same elements as the original. The elements themselves are
not copied.

If you want to write your own destructive list manipulation words,
you can use \texttt{set-car ( value cons -{}- )} and \texttt{set-cdr
( value cons -{}- )} to modify individual cons cells. Some words that
are not destructive on their inputs nonetheless create intermediate
lists which are operated on using these words. One example is \texttt{clone-list}
itself.


\section{\label{sub:Vectors}Vectors}

A \emph{vector} is a contiguous chunk of memory cells which hold references to arbitrary
objects. Vectors have the following literal syntax:

\begin{alltt}
\{ f f f t t f t t -6 {}``Hey'' \}
\end{alltt}
Use of vector literals in source code is discouraged, since vector
manipulation relies on side effects rather than return values, and
hence it is very easy to mess up a literal embedded in a word definition.

Vector words are found in the \texttt{vectors} vocabulary.

\subsection{Vectors versus lists}

Vectors are applicable to a different class of problems than lists.
Compare the relative performance of common operations on vectors and
lists:

\begin{tabular}{|r|l|l|}
\hline 
&
Lists&
Vectors\tabularnewline
\hline
\hline 
Random access of an index&
linear time&
constant time\tabularnewline
\hline 
Add new element at start&
constant time&
linear time\tabularnewline
\hline 
Add new element at end&
linear time&
constant time\tabularnewline
\hline
\end{tabular}

When using vectors, you need to pass around a vector and an index
-- when working with lists, often only a list head is passed around.
For this reason, if you need a sequence for iteration only, a list
is a better choice because the list vocabulary contains a rich collection
of recursive words.

On the other hand, when you need to maintain your own {}``stack''-like
collection, a vector is the obvious choice, since most pushes and
pops can then avoid allocating memory.

Vectors and lists can be converted back and forth using the \texttt{vector>list}
word \texttt{( vector -{}- list )} and the \texttt{list>vector} word
\texttt{( list -{}- vector )}.


\subsection{Working with vectors}

\texttt{<vector> ( capacity -{}- vector )} pushes a zero-length vector.
Storing more elements than the initial capacity grows the vector.

\texttt{vector-nth ( index vector -{}- obj )} pushes the object stored
at a zero-based index of a vector:

\begin{alltt}
0 \{ "zero" "one" \} vector-nth .
\emph{"zero"}
2 \{ 1 2 \} vector-nth .
\emph{ERROR: Out of bounds}
\end{alltt}
\texttt{set-vector-nth ( obj index vector -{}- )} stores a value into
a vector:%
\footnote{The words \texttt{get} and \texttt{set} used in this example will
be formally introduced later.%
}

\begin{alltt}
\{ "math" "CS" \} "v" set
1 "philosophy" "v" get set-vector-nth
"v" get .
\emph{\{ "math" "philosophy" \}}
4 "CS" "v" get set-vector-nth
"v" get .
\emph{\{ "math" "philosophy" f f "CS" \}}
\end{alltt}
\texttt{vector-length ( vector -{}- length )} pushes the number of
elements in a vector. As the previous two examples demonstrate, attempting
to fetch beyond the end of the vector will raise an error, while storing
beyond the end will grow the vector as necessary.

\texttt{set-vector-length ( length vector -{}- )} resizes a vector.
If the new length is larger than the current length, the vector grows
if necessary, and the new cells are filled with \texttt{f}.

\texttt{vector-push ( obj vector -{}- )} adds an object at the end
of the vector. This increments the vector's length by one.

\texttt{vector-pop ( vector -{}- obj )} removes the object at the
end of the vector and pushes it. This decrements the vector's length
by one.

The \texttt{vector-push} and \texttt{vector-pop} words can be used to implement additional stacks. For example:

\begin{alltt}
20 <vector> "state-stack" set
: push-state ( obj -- ) "state-stack" get vector-push ;
: pop-state ( -- obj ) "state-stack" get vector-pop ;
12 push-state
4 push-state
pop-state .
\emph{4}
0 push-state
pop-state .
\emph{0}
pop-state .
\emph{12}
\end{alltt}

\subsection{Vector combinators}

A pair of combinators for iterating over vectors are provided in the \texttt{vectors} vocabulary. The first is the \texttt{vector-each} word that does nothing other than applying a quotation to each element. The second is the \texttt{vector-map} word that also collects the return values of the quotation into a new vector.

\texttt{vector-each ( vector quot -{}- )} pushes each element of the vector in turn, and executes the quotation. The quotation should have a stack effect of \texttt{( obj -- )}. The vector and the quotation are not on the stack when the quotation is executed. This allows the quotation to use values below the vector for accumilation and so on.

The \texttt{stack>list} word makes use of \texttt{vector-each} to construct a list containing all elements of a given vector, in reverse order. In fact, its definition looks exactly like that of \texttt{reverse} except the \texttt{vector-each} combinator is used in place of \texttt{each}:

\begin{alltt}
: stack>list ( vector -- list )
    {[} {]} swap {[} swons {]} vector-each ;
\end{alltt}

The \texttt{vector>list} word is defined as first creating a list of all elements in the vector in reverse order using \texttt{stack>list}, and then reversing this list:

\begin{alltt}
: vector>list ( vector -- list )
    stack>list nreverse ;
\end{alltt}

\texttt{vector-map ( vector quot -{}- str )} is similar to \texttt{vector-each}, except after each iteration the return value of the quotation is collected into a new vector. The quotation should have a stack effect of \texttt{( obj -- obj )}.

The \texttt{clone-vector} word is implemented as a degenerate case of \texttt{vector-map} -- the elements of the original vector are copied into a new vector without any modification:

\begin{alltt}
: clone-vector ( vector -- vector )
    {[} {]} vector-map ;
\end{alltt}

\section{Strings}

A \emph{string} is a sequence of 16-bit Unicode characters (conventionally,
in the UTF16 encoding). Strings are input by enclosing them in quotes:

\begin{alltt}
"GET /index.html HTTP/1.0"
\end{alltt}
String literals must not span more than one line. The following is
not valid:

\begin{alltt}
"Content-Type: text/html
Content-Length: 1280"
\end{alltt}
Instead, the newline must be represented using an escape, rather than
literally. The newline escape is \texttt{\textbackslash{}n}, so we
can write:

\begin{alltt}
"Content-Type: text/html\textbackslash{}nContent-Length: 1280"
\end{alltt}
Other special characters, such as quotes and tabs can be input in
a similar manner. Here is the full list of supported character escapes:

\begin{tabular}{|r|l|}
\hline 
Character&
Escape\tabularnewline
\hline
\hline 
Quote&
\texttt{\textbackslash{}''}\tabularnewline
\hline 
Newline&
\texttt{\textbackslash{}n}\tabularnewline
\hline 
Carriage return&
\texttt{\textbackslash{}r}\tabularnewline
\hline 
Horizontal tab&
\texttt{\textbackslash{}t}\tabularnewline
\hline 
Terminal escape&
\texttt{\textbackslash{}e}\tabularnewline
\hline 
Zero chacater&
\texttt{\textbackslash{}0}\tabularnewline
\hline 
Arbitrary Unicode character&
\texttt{\textbackslash{}u}\texttt{\emph{nnnn}}\tabularnewline
\hline
\end{tabular}

The last row shows a notation for inputting any possible character
using its hexadecimal value. For example, a space character can also
be input as \texttt{\textbackslash{}u0020}.

There is no specific character data type in Factor. When characters
are extracted from a string, they are pushed on the stack as integers.
It is possible to input an integer with a value equal to that of a
Unicode character using the following special notation:

\begin{alltt}
CHAR: A .
\emph{65}
CHAR: A 1 + CHAR: B = .
\emph{t}
\end{alltt}

\subsection{Working with strings}

String words are found in the \texttt{strings} vocabulary. String
manipulation words always return a new copy of a string rather than
modifying the string in-place. Notice the absence of words such as
\texttt{set-str-nth} and \texttt{set-str-length}. Unlike lists, for
which both constructive and destuctive manipulation words are provided,
destructive string operations are only done with a distinct string
buffer type which is the topic of the next section.

\texttt{str-length ( str -{}- n )} pushes the length of a string:

\begin{alltt}
"Factor" str-length .
\emph{6}
\end{alltt}
\texttt{str-nth ( n str -{}- ch )} pushes the character located by
a zero-based index. A string is essentially a vector specialized for
storing one data type, the 16-bit unsigned character. These are returned
as integers, so printing will not yield the actual character:
\begin{alltt}
0 " " str-nth .
\emph{32}
\end{alltt}
\texttt{index-of ( str substr -{}- n )} searches a string for the
first occurrence of a substring or character. If an occurrence was
found, its index is pushed. Otherwise, -1 is pushed:

\begin{alltt}
"www.sun.com" CHAR: . index-of .
\emph{3}
"mailto:billg@microsoft.com" CHAR: / index-of .
\emph{-1}
"www.lispworks.com" ".com" index-of .
\emph{13}
\end{alltt}
\texttt{index-of{*} ( n str substr -{}- n )} works like \texttt{index-of},
except it takes a start index as an argument.

\texttt{substring ( start end str -{}- substr )} extracts a range
of characters from a string into a new string.

\texttt{split ( str split -{}- list )} pushes a new list of strings
which are substrings of the original string, taken in between occurrences
of the split string:

\begin{alltt}
"fixnum bignum ratio" " " split .
\emph{{[} "fixnum" "bignum" "ratio" {]}}
"/usr/bin/X" CHAR: / split .
\emph{{[} "" "usr" "bin" "X" {]}}
\end{alltt}
If you wish to concatenate a fixed number of strings at the top of
the stack, you can use a member of the \texttt{cat} family of words
from the \texttt{strings} vocabulary. They concatenate strings in
the order that they appear in the stack effect.

\begin{tabular}{|c|c|}
\hline 
Word&
Stack effect\tabularnewline
\hline
\hline 
\texttt{cat2}&
\texttt{( s1 s2 -{}- str )}\tabularnewline
\hline 
\texttt{cat3}&
\texttt{( s1 s2 s3 -{}- str )}\tabularnewline
\hline 
\texttt{cat4}&
\texttt{( s1 s2 s3 s4 -{}- str )}\tabularnewline
\hline 
\texttt{cat5}&
\texttt{( s1 s2 s3 s4 s5 -{}- str )}\tabularnewline
\hline
\end{tabular}

\texttt{cat ( list -{}- str )} is a generalization of the above words;
it concatenates each element of a list into a new string.

Some straightfoward examples:

\begin{alltt}
"How are you, " "Chuck" "?" cat3 .
\emph{"How are you, Chuck?"}
"/usr/bin/X" CHAR: / split cat .
\emph{"usrbinX"}
\end{alltt}
String buffers, described in the next section, provide a more flexible
means of concatenating strings.


\subsection{String buffers}

A \emph{string buffer} is a mutable string. The canonical use for
a string buffer is to combine several strings into one. This is done
by creating a new string buffer, appending strings and characters,
and finally turning the string buffer into a string.

\texttt{<sbuf> ( capacity -{}- sbuf )} pushes a new string buffer
that is capable of holding up to the specified capacity before growing.

\texttt{sbuf-append ( str/ch sbuf -{}- )} appends a string or a character
to the end of the string buffer. If an integer is given, its least significant
16 bits are interpreted as a character value:

\begin{alltt}
100 <sbuf> "my-sbuf" set
"Testing" "my-sbuf" get sbuf-append
32 "my-sbuf" get sbuf-append
\end{alltt}
\texttt{sbuf>str ( sbuf -{}- str )} pushes a string with the same
contents as the string buffer:

\begin{alltt}
"my-sbuf" get sbuf>str .
"Testing "
\end{alltt}
While usually string buffers are only used to concatenate a series
of strings, they also support the same operations as vectors.

\texttt{sbuf-nth ( n sbuf -{}- ch )} pushes the character stored at
a zero-based index of a string buffer:

\begin{alltt}
2 "A string." str-nth .
\emph{115}
\end{alltt}
\texttt{set-sbuf-nth ( ch n sbuf -{}- )} sets the character stored
at a zero-based index of a string buffer. Only the least significant
16 bits of the charcter are stored into the string buffer.

\texttt{sbuf-length ( sbuf -{}- n )} pushes the number of characters
in a string buffer. This is not the same as the capacity of the string
buffer -- the capacity is the internal storage size of the string
buffer, the length is a possibly smaller number indicating how much
storage is in use.

\texttt{set-sbuf-length ( n sbuf -{}- )} changes the length of the
string buffer. The string buffer's storage grows if necessary, and
new character positions are automatically filled with zeroes.


\subsection{String constructors}

The string construction words provide an alternative way to build up a string. Instead of passing a string buffer around on the stack, they store the string buffer in a variable. This reduces the number
of stack elements that have to be juggled.

The word \texttt{<\% ( -{}- )} begins string construction. The word
definition creates a string buffer. Instead of leaving the string
buffer on the stack, the word creates and pushes a scope on the name
stack.

The word \texttt{\% ( str/ch -{}- )} appends a string or a character
to the partial list. The word definition calls \texttt{sbuf-append}
on a string buffer located by searching the name stack.

The word \texttt{\%> ( -{}- str )} pushes the complete list. The word
definition pops the name stack and calls \texttt{sbuf>str} on the
appropriate string buffer.

Compare the following two examples -- both define a word that concatenates together all elements of a list of strings. The first one uses a string buffer stored on the stack, the second uses string construction words:

\begin{alltt}
: cat ( list -- str )
    100 <sbuf> swap {[} over sbuf-append {]} each sbuf>str ;

: cat ( list -- str )
    <\% {[} \% {]} each \%> ;
\end{alltt}

The scope created by \texttt{<\%} and \texttt{\%>} is \emph{dynamic}; that is, all code executed between two words is part of the scope. This allows the call to \texttt{\%} to occur in a nested word. For example, here is a pair of definitions that turn an association list of strings into a string of the form \texttt{key1=value1 key2=value2 ...}:

\begin{alltt}
: pair\% ( pair -{}- )
    unswons \% "=" \% \% ;

: assoc>string ( alist -{}- )
    <\% [ pair\% " " \% ] each \%> ;
\end{alltt}

\subsection{String combinators}

A pair of combinators for iterating over strings are provided in the \texttt{strings} vocabulary. The first is the \texttt{str-each} word that does nothing other than applying a quotation to each character. The second is the \texttt{str-map} word that also collects the return values of the quotation into a new string.

\texttt{str-each ( str quot -{}- )} pushes each character of the string in turn, and executes the quotation. The quotation should have a stack effect of \texttt{( ch -{}- )}. The string and the quotation are not on the stack when the quotation is executed. This allows the quotation to use values below the string for accumilation and so on. The following example counts the number of occurrences of the letter ``a'' in a string:

\begin{alltt}
: count-a ( str -- n )
    0 swap {[} CHAR: a = {[} 1 + {]} when {]} str-each ;

"Lets just say that you may stay" count-a .
\emph{4}
\end{alltt}

\texttt{str-map ( str quot -{}- str )} is similar to \texttt{str-each}, except after each iteration the return value of the quotation is collected into a new string. The quotation should have a stack effect of \texttt{( ch -- str/ch )}. The following example replaces all occurrences of the space character in the string with \texttt{+}:

\begin{alltt}
"We do not like spaces" {[} CHAR: \textbackslash{}s CHAR: + replace {]} str-map .
\emph{"We+do+not+like+spaces"}
\end{alltt}

\subsection{Printing and reading strings}

The following two words from the \texttt{stdio} vocabulary output text to the terminal. They differ from \texttt{.}
in that they print strings only, without surrounding quotes, and raise
an error when given any other data type. The word \texttt{.} prints any Factor
object in a form suited for parsing, hence it quotes strings.

\texttt{write ( str -{}- )} writes a string to the standard output
device, without a terminating newline.

\texttt{print ( str -{}- )} writes a string followed by a newline
character. To print a single newline character, use \texttt{terpri (
-{}- )} instead of passing a blank string to \texttt{print}.

Input can be read from the terminal, a line at a time.

\texttt{read ( -{}- str )} reads a line of input from the standard
input device, terminated by a newline.

\begin{alltt}
"a" write "b" write
ab
{[} "hello" "world" {]} {[} print {]} each
hello
world
\end{alltt}
Often a string representation of a number, usually one read from an
input source, needs to be turned into a number. Unlike some languages,
in Factor the conversion from a string such as {}``123'' into the
number 123 is not automatic. To turn a string into a number, use one
of two words in the \texttt{parser} vocabulary.

\texttt{str>number ( str -{}- n )} creates an integer, ratio or floating
point literal from its string representation. If the string does not
reprent a valid number, an exception is thrown.

\texttt{parse-number ( str -{}- n/f )} pushes \texttt{f} on failure, rather
than raising an exception.

\texttt{unparse ( n -{}- str )} pushes the string representation of
a number.


\section{PRACTICAL: Contractor timesheet}

For the second practical example, we will code a small program that tracks how long you spend working on tasks. It will provide two primary functions, one for adding a new task and measuring how long you spend working on it, and another to print out the timesheet. A typical interaction looks like this:

\begin{alltt}
timesheet-app
\emph{
(E)xit
(A)dd entry
(P)rint timesheet

Enter a letter between ( ) to execute that action.}
a
\emph{Start work on the task now. Press ENTER when done.

Please enter a description:
Working on the Factor HTTP server

(E)xit
(A)dd entry
(P)rint timesheet

Enter a letter between ( ) to execute that action.}
a
\emph{Start work on the task now. Press ENTER when done.

Please enter a description:}
Writing a kick-ass web app
\emph{
(E)xit
(A)dd entry
(P)rint timesheet

Enter a letter between ( ) to execute that action.}
p
\emph{TIMESHEET:
Working on the Factor HTTP server                           0:25
Writing a kick-ass web app                                  1:03

(E)xit
(A)dd entry
(P)rint timesheet

Enter a letter between ( ) to execute that action.}
x
\end{alltt}

Once you have finished working your way through this tutorial, you might want to try extending the program -- for example, it could print the total hours, prompt for an hourly rate, then print the amount of money that should be billed.

\subsection{Measuring a duration of time}

When you begin working on a new task, you tell the timesheet you want
to add a new entry. It then measures the elapsed time until you specify
the task is done, and prompts for a task description.

The first word we will write is \texttt{measure-duration}. We measure
the time duration by using the \texttt{millis} word \texttt{( -{}-
m )} to take the time before and after a call to \texttt{read}. The
\texttt{millis} word pushes the number of milliseconds since a certain
epoch -- the epoch does not matter here since we are only interested
in the difference between two times.

A first attempt at \texttt{measure-duration} might look like this:

\begin{alltt}
: measure-duration millis read drop millis - ;
measure-duration .
\end{alltt}

This word definition has the right general idea, however, the result
is negative. Also, we would like to measure durations in minutes,
not milliseconds:

\begin{alltt}
: measure-duration ( -{}- duration )
    millis
    read drop
    millis swap - 1000 /i 60 /i ;
\end{alltt}

Note that the \texttt{/i} word \texttt{( x y -{}- x/y )}, from the
\texttt{math} vocabulary, performs truncating division. This
makes sense, since we are not interested in fractional parts of a
minute here.

\subsection{Adding a timesheet entry}

Now that we can measure a time duration at the keyboard, lets write
the \texttt{add-entry-prompt} word. This word does exactly what one
would expect -- it prompts for the time duration and description,
and leaves those two values on the stack:

\begin{alltt}
: add-entry-prompt ( -{}- duration description )
    "Start work on the task now. Press ENTER when done." print
    measure-duration
    "Please enter a description:" print
    read ;
\end{alltt}

You should interactively test this word. Measure off a minute or two,
press ENTER, enter a description, and press ENTER again. The stack
should now contain two values, in the same order as the stack effect
comment.

Now, almost all the ingredients are in place. The final add-entry
word calls add-entry-prompt, then pushes the new entry on the end
of the timesheet vector:

\begin{alltt}
: add-entry ( timesheet -{}- )
    add-entry-prompt cons swap vector-push ;
\end{alltt}

Recall that timesheet entries are cons cells where the car is the
duration and the cdr is the description, hence the call to \texttt{cons}.
Note that this word side-effects the timesheet vector. You can test
it interactively like so:

\begin{alltt}
10 <vector> dup add-entry
\emph{Start work on the task now. Press ENTER when done.}
\emph{Please enter a description:}
\emph{Studying Factor}
.
\emph{\{ {[} 2 | "Studying Factor" {]} \}}
\end{alltt}

\subsection{Printing the timesheet}

The hard part of printing the timesheet is turning the duration in
minutes into a nice hours/minutes string, like {}``01:15''. We would
like to make a word like the following:

\begin{alltt}
135 hh:mm .
\emph{01:15}
\end{alltt}

First, we can make a pair of words hh and mm to extract the hours
and minutes, respectively. This can be achieved using truncating division,
and the modulo operator -- also, since we would like strings to be
returned, the \texttt{unparse} word \texttt{( obj -{}- str )} from
the \texttt{unparser} vocabulary is called to turn the integers into
strings:

\begin{alltt}
: hh ( duration -{}- str ) 60 /i unparse ;
: mm ( duration -{}- str ) 60 mod unparse ;
\end{alltt}

The \texttt{hh:mm} word can then be written, concatenating the return
values of \texttt{hh} and \texttt{mm} into a single string using string
construction:

\begin{alltt}
: hh:mm ( millis -{}- str ) <\% dup hh \% ":" \% mm \% \%> ;
\end{alltt}
However, so far, these three definitions do not produce ideal output.
Try a few examples:

\begin{alltt}
120 hh:mm .
2:0
130 hh:mm .
2:10
\end{alltt}
Obviously, we would like the minutes to always be two digits. Luckily,
there is a \texttt{digits} word \texttt{( str n -{}- str )} in the
\texttt{format} vocabulary that adds enough zeros on the left of the
string to give it the specified length. Try it out:

\begin{alltt}
{}``23'' 2 digits .
\emph{{}``23''}
{}``7'' 2 digits .
\emph{{}``07''}
\end{alltt}
We can now change the definition of \texttt{mm} accordingly:

\begin{alltt}
: mm ( duration -{}- str ) 60 mod unparse 2 digits ;
\end{alltt}
Now that time duration output is done, a first attempt at a definition
of \texttt{print-timesheet} looks like this:

\begin{alltt}
: print-timesheet ( timesheet -{}- )
    {[} uncons write ": " write hh:mm print {]} vector-each ;
\end{alltt}
This works, but produces ugly output:

\begin{alltt}
\{ {[} 30 | "Studying Factor" {]} {[} 65 | "Paperwork" {]} \}
print-timesheet
\emph{Studying Factor: 0:30}
\emph{Paperwork: 1:05}
\end{alltt}
It would be much nicer if the time durations lined up in the same
column. First, lets factor out the body of the \texttt{vector-each}
loop into a new \texttt{print-entry} word before it gets too long:

\begin{alltt}
: print-entry ( duration description -{}- )
    write {}``: '' write hh:mm print ;

: print-timesheet ( timesheet -{}- )
    {[} uncons print-entry {]} vector-each ;
\end{alltt}
We can now make \texttt{print-entry} line up columns using the \texttt{pad-string}
word \texttt{( str n -{}- str )}.

\begin{alltt}
: print-entry ( duration description -{}- )
    dup
    write
    50 swap pad-string write 
    hh:mm print ;
\end{alltt}
In the above definition, we first print the description, then enough
blanks to move the cursor to column 60. So the description text is
left-justified. If we had interchanged the order of the second and
third line in the definition, the description text would be right-justified.

Try out \texttt{print-timesheet} again, and marvel at the aligned
columns:

\begin{alltt}
\{ {[} 30 | "Studying Factor" {]} {[} 65 | "Paperwork" {]} \}
print-timesheet
\emph{Studying Factor                                   0:30}
\emph{Paperwork                                         1:05}
\end{alltt}

\subsection{The main menu}

Finally, we will code a main menu that looks like this:

\begin{alltt}

(E)xit
(A)dd entry
(P)rint timesheet

Enter a letter between ( ) to execute that action.
\end{alltt}

We will represent the menu as an association list. Recall that an association list is a list of pairs, where the car of each pair is a key, and the cdr is a value. Our keys will literally be keyboard keys (``e'', ``a'' and ``p''), and the values will themselves be pairs consisting of a menu item label and a quotation.

The first word we will code is \texttt{print-menu}. It takes an association list, and prints the second element of each pair's value. Note that \texttt{terpri} simply prints a blank line:

\begin{alltt}
: print-menu ( menu -{}- )
    terpri {[} cdr car print {]} each terpri
    "Enter a letter between ( ) to execute that action." print ;
\end{alltt}

You can test \texttt{print-menu} with a short association list:

\begin{alltt}
{[} {[} "x" "(X)yzzy" 2 2 + . {]} {[} "f" "(F)oo" -1 sqrt . {]} {]} print-menu
\emph{
Xyzzy
Foo

Enter a letter between ( ) to execute that action.}
\end{alltt}

The next step is to write a \texttt{menu-prompt} word that takes the same association list, reads a line of input from the keyboard, and executes the quotation associated with that line. Recall that the \texttt{assoc} word returns \texttt{f} if the specified key could not be found in the association list. The below definition makes use of a conditional to signal an error in that case:

\begin{alltt}
: menu-prompt ( menu -{}- )
    read swap assoc dup {[}
        cdr call
    {]} {[}
        "Invalid input: " swap unparse cat2 throw
    {]} ifte ;
\end{alltt}

Try applying the new \texttt{menu-prompt} word to the association list we used to test \texttt{print-menu}. You should verify that entering \texttt{x} causes the quotation \texttt{{[} 2 2 + . {]}} to be executed:

\begin{alltt}
{[} {[} "x" "(X)yzzy" 2 2 + . {]} {[} "f" "(F)oo" -1 sqrt . {]} {]} menu-prompt
x
\emph{4}
\end{alltt}

Finally, we want a \texttt{menu} word that first prints a menu, then prompts for and acts on input:

\begin{alltt}
: menu ( menu -{}- )
    dup print-menu menu-prompt ;
\end{alltt}

Considering the stack effects of \texttt{print-menu} and \texttt{menu-prompt}, it should be obvious why the \texttt{dup} is needed.

\subsection{Finishing off}

We now need a \texttt{main-menu} word. It takes the timesheet vector from the stack, and recursively calls itself until the user requests that the timesheet application exits:

\begin{alltt}
: main-menu ( timesheet -{}- )
    {[}
        {[} "e" "(E)xit" drop {]}
        {[} "a" "(A)dd entry" dup add-entry main-menu {]}
        {[} "p" "(P)rint timesheet" dup print-timesheet main-menu {]}
    {]} menu ;
\end{alltt}

Note that unless the first option is selected, the timesheet vector is eventually passed into the recursive \texttt{main-menu} call.

All that remains now is the ``main word'' that runs the program with an empty timesheet vector. Note that the initial capacity of the vector is 10 elements, however this is not a limit -- adding more than 10 elements will grow the vector:

\begin{alltt}
: timesheet-app ( -{}- )
    10 <vector> main-menu ;
\end{alltt}

\subsection{The complete program}

\begin{verbatim}
! Contractor timesheet example

IN: timesheet
USE: combinators
USE: errors
USE: format
USE: kernel
USE: lists
USE: math
USE: parser
USE: stack
USE: stdio
USE: strings
USE: unparser
USE: vectors

! Adding a new entry to the time sheet.

: measure-duration ( -- duration )
    millis
    read drop
    millis swap - 1000 /i 60 /i ;

: add-entry-prompt ( -- duration description )
    "Start work on the task now. Press ENTER when done." print
    measure-duration
    "Please enter a description:" print
    read ;

: add-entry ( timesheet -- )
    add-entry-prompt cons swap vector-push ;

! Printing the timesheet.

: hh ( duration -- str ) 60 /i ;
: mm ( duration -- str ) 60 mod unparse 2 digits ;
: hh:mm ( millis -- str ) <% dup hh % ":" % mm % %> ;

: print-entry ( duration description -- )
    dup write
    60 swap pad-string write
    hh:mm print ;

: print-timesheet ( timesheet -- )
    "TIMESHEET:" print
    [ uncons print-entry ] vector-each ;

! Displaying a menu

: print-menu ( menu -- )
    terpri [ cdr car print ] each terpri
    "Enter a letter between ( ) to execute that action." print ;

: menu-prompt ( menu -- )
    read swap assoc dup [
        cdr call
    ] [
        "Invalid input: " swap unparse cat2 throw
    ] ifte ;

: menu ( menu -- )
    dup print-menu menu-prompt ;

! Main menu

: main-menu ( timesheet -- )
    [
        [ "e" "(E)xit" drop ]
        [ "a" "(A)dd entry" dup add-entry main-menu ]
        [ "p" "(P)rint timesheet" dup print-timesheet main-menu ]
    ] menu ;

: timesheet-app ( -- )
    10 <vector> main-menu ;
\end{verbatim}

\section{Object orientation}

\subsection{Identity and equality}

The previously-mentioned \texttt{=} word in the \texttt{kernel} vocabulary, as well as the \texttt{assoc}, \texttt{contains} and \texttt{unique} words in the \texttt{lists} vocabulary all rely on object equality as part of their operation.

What does it mean for two objects to be ``equal''? In actual fact, there are two ways of comparing objects. Two object references can be compared for \emph{identity} using the \texttt{eq? ( obj obj -{}- ? )} word. This only returns true if both references point to the same object. A weaker form of comparison is the \texttt{= ( obj obj -{}- ? )} word, which checks if two objects ``have the same shape''.
If two objects are \texttt{eq?}, they will also be \texttt{=}.

For example, two literal objects with the same printed representation are as a general rule not always \texttt{eq?}, however they are \texttt{=}:

\begin{alltt}
{[} 1 2 3 {]} {[} 1 2 3 {]} eq? .
\emph{f}
{[} 1 2 3 {]} {[} 1 2 3 {]} = .
\emph{t}
\end{alltt}

On the other hand, duplicating an object reference on the stack using \texttt{dup} or similar, will give two references which are \texttt{eq?}:

\begin{alltt}
"Hello" dup eq? .
\emph{t}
\end{alltt}

An object can be cloned using \texttt{clone ( obj -{}- obj )}. The clone will no longer be \texttt{eq?} to the original (unless the original is immutable, in which case cloning is a no-op); however clones are always \texttt{=}.

\subsection{Hashtables}

A hashtable, much like an association list, stores key/value pairs, and offers lookup by key. However, whereas an association list must be searched linearly to locate keys, a hashtable uses a more sophisticated method. Key/value pairs are sorted into \emph{buckets} using a \emph{hash function}. If two objects are equal, then they must have the same hash code; but not necessarily vice versa. To look up the value associated with a key, only the bucket corresponding to the key has to be searched. A hashtable is simply a vector of buckets, where each bucket is an association list.

\texttt{<hashtable> ( capacity -{}- hash )} creates a new hashtable with the specified number of buckets. A hashtable with one bucket is basically an association list. Right now, a ``large enough'' capacity must be specified, and performance degrades if there are too many key/value pairs per bucket. In a future implementation, hashtables will grow as needed as the number of key/value pairs increases.

\texttt{hash ( key hash -{}- value )} looks up the value associated with a key in the hashtable. Pushes \texttt{f} if no pair with this key is present. Note that \texttt{hash} cannot differentiate between a key that is not present at all, or a key with a value of \texttt{f}.

\texttt{hash* ( key hash -{}- {[} key | value {]})} looks for
a pair with this key, and pushes the pair itself. Unlike \texttt{hash},
\texttt{hash{*}} returns different values in the cases of a value
set to \texttt{f}, or an undefined value.

\texttt{set-hash ( value key hash -{}- )} stores a key/value pair in a hashtable.

examples, and hash>alist, alist>hash, hash-keys, hash-values

\subsection{Variables}

Notice that until now, all the code except a handful of examples has only used the stack for storage. You can also use variables to store temporary data, much like in other languages, however their use is not so prevalent. This is not a coincidence -- Fator was designed this way, and mastery of the stack is essential. Using variables where the stack is more appropriate leads to ugly, unreusable code.

Variables are typically used for longer-term storage of data, and for temporary storage of objects that are being constructed, where using the stack would be ackward. Another use for variables is compound data structures, realized as nested namespaces of variables. This concept should be instantly familiar to anybody who's used an object-oriented programming language.

The words \texttt{get ( name -{}- value )} and \texttt{set ( value name -{}- )} retreive and store variable values, respectively. Variable names are strings, and they do not have to be declared before use. For example:

\begin{alltt}
5 "x" set
"x" get .
\emph{5}
\end{alltt}

\subsection{Namespaces}

Only having one list of variable name/value bindings would make the language terribly inflexible. Instead, a variable has any number of potential values, one per namespace. There is a notion of a ``current namespace''; the \texttt{set} word always stores variables in the current namespace. On the other hand, \texttt{get} traverses up the stack of namespace bindings until it finds a variable with the specified name.

\texttt{bind ( namespace quot -{}- )} executes a quotation in the dynamic scope of a namespace. For example, the following sets the value of \texttt{x} to 5 in the global namespace, regardless of the current namespace at the time the word was called.

\begin{alltt}
: global-example ( -- )
    global {[} 5 "x" set {]} bind ;
\end{alltt}

\texttt{<namespace> ( -{}- namespace )} creates a new namespace object. Actually, a namespace is just a hashtable, with a default capacity.

\texttt{with-scope ( quot -{}- )} combines \texttt{<namespace>} with \texttt{bind} by executing a quotation in a new namespace.

get example

describe

\subsection{The name stack}

So far, we have seen what we called ``the stack'' store intermediate values between computations. In fact Factor maintains a number of other stacks, and the formal name for the stack we've been dealing with so far is the \emph{data stack}.

Another stack is the \emph{call stack}. When a colon definition is invoked, the position within the current colon definition is pushed on the stack. This ensures that calling words return to the caller, just as in any other language with subroutines.\footnote{Factor supports a variety of structures for implementing non-local word exits, such as exceptions, co-routines, continuations, and so on. They all rely on manipulating the call stack and are described in later sections.}

The \emph{name stack} is the focus of this section. The \texttt{bind} combinator creates dynamic scope by pushing and popping namespaces on the name stack. Its definition is simpler than one would expect:

\begin{alltt}
: bind ( namespace quot -- )
    swap >n call n> drop ;
\end{alltt}

The words \texttt{>n} and \texttt{n>} push and pop the name stack, respectively. Observe the stack flow in the definition of \texttt{bind}; the namespace goes on the name stack, the quotation is called, and the name space is popped and discarded.

The name stack is really just a vector. The words \texttt{>n} and \texttt{n>} are implemented as follows:

\begin{alltt}
: >n ( namespace -- n:namespace ) namestack* vector-push ;
: n> ( n:namespace -- namespace ) namestack* vector-pop ;
\end{alltt}

\section{Metaprogramming}

Recall that code quotations are in fact just linked lists. Factor code is data, and vice versa. Essentially, the interpreter iterates through code quotations, pushing literals and executing words. When a word is executed, one of two things happen -- either the word has a colon definition, and the interpreter is invoked recursively on the definition, or the word is primitive, and it is executed by the underlying virtual machine. A word is itself a first-class object.

It is the job of the parser to transform source code denoting literals and words into their internal representations. This is done using a vocabulary of \emph{parsing words}. The prettyprinter does the converse, by printing out data structures in a parsable form (both to humans and Factor). Because code is data, text representation of source code doubles as a way to serialize almost any Factor object.

\subsection{Looking at words}

Try pushing a list of words on the stack, and take its first element:

\begin{alltt}
{[} * + {]} car .s
\emph{\{ * \}}
\end{alltt}

What happened here? Instead of being executed, a ``naked'', unquoted word was pushed on the stack. The predicate \texttt{word? ( obj -{}- ? )} from the \texttt{words} vocabulary tests if the top of the stack is a word. Another way to get a word on the stack is to do a vocabulary search using a word name and a list of vocabularies to search in:

\begin{alltt}
"car" {[} "lists" {]} search .s
\emph{\{ car \}}
\end{alltt}

The \texttt{search} word will push \texttt{f} if the word is not defined. A new word can be created in a specified vocabulary explicitly:

\begin{alltt}
"start-server" "user" create .s
\emph{\{ start-server \}}
\end{alltt}

Two words are only ever equal under the \texttt{=} operator if they identify the same underlying object. Word objects are composed of three slots, named as follows.

\begin{tabular}{|r|l|}
\hline 
Slot&
Description\tabularnewline
\hline
\hline 
Primitive&
A number identifying a virtual machine operation.\tabularnewline
\hline 
Parameter&
An object parameter for the virtual machine operation.\tabularnewline
\hline 
Property list&
An association list of name/value pairs.\tabularnewline
\hline
\end{tabular}

If the primitive number is set to 1, the word is a colon definition and the parameter must be a quotation. Any other primitive number denotes a function of the virtual machine, and the parameter is ignored. Do not rely on primitive numbers in your code, instead use the \texttt{compound? ( obj -{}- ? )} and \texttt{primitive? ( obj -{}- ? )} predicates.

The word \texttt{define ( word quot -{}- )} defines a word to have the specified colon definition. Note that \texttt{create} and  \texttt{define} perform an action somewhat analagous to the \texttt{: ... ;} notation for colon definitions, except at parse time rather than run time.

\subsection{The prettyprinter}

We've already seen the word \texttt{.} which prints the top of the stack in a form that may be read back in. The word \texttt{prettyprint} is similar, except the output is in an indented, multiple-line format. Both words are in the \texttt{prettyprint} vocabulary. Here is an example:

\begin{alltt}
{[} 1 {[} 2 3 4 {]} 5 {]} .
\emph{{[} 1 {[} 2 3 4 {]} 5 {]}}
{[} 1 {[} 2 3 4 {]} 5 {]} prettyprint
\emph{{[}
    1 {[}
        2 3 4
    {]} 5
{]}}
\end{alltt}


\subsection{The parser}

\subsection{Parsing words}

Lets take a closer look at Factor syntax. Consider a simple expression,
and the result of evaluating it in the interactive interpreter:

\begin{alltt}
2 3 + .
\emph{5}
\end{alltt}
The interactive interpreter is basically an infinite loop. It reads
a line of input from the terminal, parses this line to produce a \emph{quotation},
and executes the quotation.

In the parse step, the input text is tokenized into a sequence of
white space-separated tokens. First, the interpreter checks if there
is an existing word named by the token. If there is no such word,
the interpreter instead treats the token as a number.%
\footnote{Of course, Factor supports a full range of data types, including strings,
lists and vectors. Their source representations are still built from
numbers and words, however.%
}

Once the expression has been entirely parsed, the interactive interpreter
executes it.

This parse time/run time distinction is important, because words fall
into two categories; {}``parsing words'' and {}``running words''.

The parser constructs a parse tree from the input text. When the parser
encounters a token representing a number or an ordinary word, the
token is simply appended to the current parse tree node. A parsing
word on the other hand is executed \emph{}immediately after being
tokenized. Since it executes in the context of the parser, it has
access to the raw input text, the entire parse tree, and other parser
structures.

Parsing words are also defined using colon definitions, except we
add \texttt{parsing} after the terminating \texttt{;}. Here are two
examples of definitions for words \texttt{foo} and \texttt{bar}, both
are identical except in the second example, \texttt{foo} is defined
as a parsing word:

\begin{alltt}
! Lets define 'foo' as a running word.
: foo "1) foo executed." print ;
: bar foo "2) bar executed." print ;
bar
\emph{1) foo executed}
\emph{2) bar executed}
bar
\emph{1) foo executed}
\emph{2) bar executed}

! Now lets define 'foo' as a parsing word.
: foo "1) foo executed." print ; parsing
: bar foo "2) bar executed." ;
\emph{1) foo executed}
bar
\emph{2) bar executed}
bar
\emph{2) bar executed}
\end{alltt}
In fact, the word \texttt{{}''} that denotes a string literal is
a parsing word -- it reads characters from the input text until the
next occurrence of \texttt{{}''}, and appends this string to the
current node of the parse tree. Note that strings and words are different
types of objects. Strings are covered in great detail later.


\section{PRACTICAL: Infix syntax}


\section{Continuations}

Call stack how it works and >r/r>

Generators, co-routines, multitasking, exception handling


\section{HTTP Server}


\section{PRACTICAL: Some web app}
\end{document}
